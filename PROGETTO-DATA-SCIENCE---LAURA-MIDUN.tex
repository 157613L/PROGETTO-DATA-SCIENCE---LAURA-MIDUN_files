% Options for packages loaded elsewhere
\PassOptionsToPackage{unicode}{hyperref}
\PassOptionsToPackage{hyphens}{url}
%
\documentclass[
]{article}
\usepackage{amsmath,amssymb}
\usepackage{lmodern}
\usepackage{iftex}
\ifPDFTeX
  \usepackage[T1]{fontenc}
  \usepackage[utf8]{inputenc}
  \usepackage{textcomp} % provide euro and other symbols
\else % if luatex or xetex
  \usepackage{unicode-math}
  \defaultfontfeatures{Scale=MatchLowercase}
  \defaultfontfeatures[\rmfamily]{Ligatures=TeX,Scale=1}
\fi
% Use upquote if available, for straight quotes in verbatim environments
\IfFileExists{upquote.sty}{\usepackage{upquote}}{}
\IfFileExists{microtype.sty}{% use microtype if available
  \usepackage[]{microtype}
  \UseMicrotypeSet[protrusion]{basicmath} % disable protrusion for tt fonts
}{}
\makeatletter
\@ifundefined{KOMAClassName}{% if non-KOMA class
  \IfFileExists{parskip.sty}{%
    \usepackage{parskip}
  }{% else
    \setlength{\parindent}{0pt}
    \setlength{\parskip}{6pt plus 2pt minus 1pt}}
}{% if KOMA class
  \KOMAoptions{parskip=half}}
\makeatother
\usepackage{xcolor}
\IfFileExists{xurl.sty}{\usepackage{xurl}}{} % add URL line breaks if available
\IfFileExists{bookmark.sty}{\usepackage{bookmark}}{\usepackage{hyperref}}
\hypersetup{
  pdftitle={Progetto data science - Laura Midun},
  hidelinks,
  pdfcreator={LaTeX via pandoc}}
\urlstyle{same} % disable monospaced font for URLs
\usepackage[margin=1in]{geometry}
\usepackage{graphicx}
\makeatletter
\def\maxwidth{\ifdim\Gin@nat@width>\linewidth\linewidth\else\Gin@nat@width\fi}
\def\maxheight{\ifdim\Gin@nat@height>\textheight\textheight\else\Gin@nat@height\fi}
\makeatother
% Scale images if necessary, so that they will not overflow the page
% margins by default, and it is still possible to overwrite the defaults
% using explicit options in \includegraphics[width, height, ...]{}
\setkeys{Gin}{width=\maxwidth,height=\maxheight,keepaspectratio}
% Set default figure placement to htbp
\makeatletter
\def\fps@figure{htbp}
\makeatother
\setlength{\emergencystretch}{3em} % prevent overfull lines
\providecommand{\tightlist}{%
  \setlength{\itemsep}{0pt}\setlength{\parskip}{0pt}}
\setcounter{secnumdepth}{-\maxdimen} % remove section numbering
\ifLuaTeX
  \usepackage{selnolig}  % disable illegal ligatures
\fi

\title{Progetto data science - Laura Midun}
\author{}
\date{\vspace{-2.5em}2022-05-27}

\begin{document}
\maketitle

\#install.packages(``readxl'')

\#library(``readxl'')

\#LIBRERIE USATE\\
library(tidyr) library (dplyr) library(ggplot2) library(magrittr)
\#???????

\#AGGIORNAMENTO LIBRERIE update.packages(``tidyr'')
update.packages(``dplyr'') update.packages(``ggplot2'')

\hypertarget{importazione-dataset}{%
\section{IMPORTAZIONE DATASET}\label{importazione-dataset}}

\#IMPORT \#setwd(``C:/Users/laura/Desktop/SCIENZE DEI DATI/RELAZIONE
DATA SCIENCE/INTERNATIONAL
TRADE/PROGETTO-DATA-SCIENCE---LAURA-MIDUN\_files/'') read.csv(``Import
(\$ thousand) 1994-2019.csv'', sep = ``;'') read.csv(``Import partner
share (\%) 1994-2019.csv'', sep = ``;'') read.csv(``Import share in
total products 1994-2019.csv'', sep = ``;'')

\#EXPORT read.csv(``Export (\$ thousand) 1994-2019.csv'', sep = ``;'')
read.csv(``Export partner share (\%) 1994-2019.csv'', sep = ``;'')
read.csv(``Export share in total products 1994-2019.csv'', sep = ``;'')

\#ASSEGNAZIONE DI UNA VARIABILE AI DATASET import = read.csv(``Import
(\$ thousand) 1994-2019.csv'', header = TRUE, sep = ``;'')
import\_partner\_share = read.csv(``Import partner share (\%)
1994-2019.csv'', header = TRUE, sep = ``;'') import\_products =
read.csv(``Import share in total products 1994-2019.csv'', header =
TRUE, sep = ``;'')

export = read.csv(``Export (\$ thousand) 1994-2019.csv'', header = TRUE,
sep = ``;'') export\_partner\_share = read.csv(``Export partner share
(\%) 1994-2019.csv'', header = TRUE, sep = ``;'') export\_products =
read.csv(``Export share in total products 1994-2019.csv'', header =
TRUE, sep = ``;'')

\#PER VISUALIZZARE LE TABELLE View(import) View(import\_partner\_share)
View(import\_products)

View(export) View(export\_partner\_share) View(export\_products)

\hypertarget{trasformare-le-colonne-in-righe}{%
\section{TRASFORMARE LE COLONNE IN
RIGHE}\label{trasformare-le-colonne-in-righe}}

\#pivot\_longer() --\textgreater{} PERMETTE DI CONVERTIRE LE COLONNE IN
RIGHE

\#IMPORT library(tidyr) library(dplyr) library(magrittr)

\hypertarget{colonne-da-considerare-reporter-name-partner-name-product-group-1994-2019}{%
\section{colonne da considerare --\textgreater{} reporter name, partner
name, product group,
1994-2019}\label{colonne-da-considerare-reporter-name-partner-name-product-group-1994-2019}}

\#import pivot\_longer( import, c(`X1994':`X2019'), names\_to =
``Year'', values\_to = ``Import.(US\$.Thousand)'' )

\hypertarget{select-seleziona-le-colonne}{%
\section{select() --\textgreater{} seleziona le
colonne}\label{select-seleziona-le-colonne}}

import \%\textgreater\% select(Reporter.Name, Partner.Name,
Product.Group, Year, Import.(US\$.Thousand))

\#import\_partner\_share pivot\_longer( import\_partner\_share,
c(`X1994':`X2019'), names\_to = ``Year'', values\_to =
``Import.Partner.Share.(\%)'' )

import\_partner\_share \%\textgreater\% select(Reporter.Name,
Partner.Name, Year, Import.Partner.Share.(\%))

\#tab1 \textless- merge(import, import\_partner\_share, by =
``Reporter.Name'')

\#import\_products pivot\_longer( import\_products, c(`X1994':`X2019'),
names\_to = ``Year'', values\_to =
``Import.Share.In.Total.Products.(\%)'' )

Import\_products \%\textgreater\% select(Reporter.Name, Partner.Name,
Year, Import.Share.In.Total.Products.(\%))

\#EXPORT

\#export pivot\_longer( export, c(`X1994':`X2019'), names\_to =
``year'', values\_to = ``Export.(US\$.Thousand)'' )

export \%\textgreater\% select(Reporter.Name, Partner.Name,
Product.Group, Year, Export.(US\$.Thousand))

\#export\_partner\_share pivot\_longer( export\_partner\_share,
c(`X1994':`X2019'), names\_to = ``year'', values\_to =
``Export.Partner.Share.(\%)'' )

export\_partner\_share \%\textgreater\% select(Reporter.Name,
Partner.Name, Year, Export.Partner.Share.(\%))

\#export\_products

pivot\_longer( export\_products, c(`X1994':`X2019'), names\_to =
``year'', values\_to = ``Export.Share.In.Total.Products.(\%)'' )

export\_products \%\textgreater\% select(Reporter.Name, Partner.Name,
Year, Export.Share.In.Total.Products.(\%))

\#RAPPRESENTAZIONE GRAFICA

\#DOMANDE DA PORSI \#-perchè in determinati anni il guadagno/la vendita
(in dollari) di merci è sceso/a? a causa di cosa? \#-quali sono stasti
gli anni in cui si può notare una diminuzione delle
importazioni/esportazioni? a causa di cosa? \#-in quali anni c'è stato
un aumento delle importazioni/esportazioni? a causa di cosa? \#-in quali
anniS c'è stato un aumento delle importazioni e delle importazioni? a
causa di cosa? \#-in quali anni c'è stato una diminuzione delle
importazioni e delle esportazioni? a causa di cosa? \#-quali sono i
partner più importanti per l'italia? (quali sono per l'italia i maggiori
paesi importatori/esportatori?)

library(ggplot2)

\#rappresentazione delle importazioni da tutti i pesi presi in
considerazione all'italia ggplot(import, aes(Year,
Import(US\$.Thousand), color = Partner.Name)) + geom\_line(alpha = 1/2,
show.legend = FALSE) + scale\_colour\_manual(values =
Partner.Name\_colors) + theme\_classic() \#+ \#ggtitle(``Import'') ???

``C:/Program Files/RStudio/bin/quarto/bin/pandoc'' +RTS -K512m -RTS
``C:/Users/laura/Desktop/SCIENZE DEI DATI/RELAZIONE DATA
SCIENCE/INTERNATIONAL
TRADE/PROGETTO-DATA-SCIENCE---LAURA-MIDUN\_files/PROGETTO-DATA-SCIENCE---LAURA-MIDUN.knit.md''
--to html4 --from
markdown+autolink\_bare\_uris+tex\_math\_single\_backslash --output
pandoc455875843453.html --lua-filter
``C:\Users\laura\Documents\R\win-library\textbackslash4.1\rmarkdown\rmarkdown\lua\pagebreak.lua''
--lua-filter
``C:\Users\laura\Documents\R\win-library\textbackslash4.1\rmarkdown\rmarkdown\lua\latex-div.lua''
--self-contained --variable bs3=TRUE --standalone --section-divs
--template
``C:\Users\laura\Documents\R\win-library\textbackslash4.1\rmarkdown\rmd\h\default.html''
--no-highlight --variable highlightjs=1 --variable theme=bootstrap
--mathjax --variable
``mathjax-url=\url{https://mathjax.rstudio.com/latest/MathJax.js?config=TeX-AMS-MML_HTMLorMML}''
--include-in-header
``C:\Users\laura\AppData\Local\Temp\RtmpAd9cMs\rmarkdown-str4558455e347b.html''

\end{document}
